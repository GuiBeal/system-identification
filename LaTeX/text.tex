\documentclass{ppgeesa}

\usepackage[hidelinks]{hyperref}
\usepackage{url}
\usepackage{breakurl}

\usepackage[detect-weight,detect-family]{siunitx}
\sisetup{output-decimal-marker = {,}}
\sisetup{exponent-product = {\cdot}}

\usepackage{amsmath}
\usepackage{amssymb}
\usepackage{mathtools}
\usepackage{resizegather}
\usepackage{icomma}
\mathtoolsset{showonlyrefs}
\newcommand{\Prod}{\,}

\usepackage{graphicx}
\usepackage{epstopdf}
\graphicspath{{../img/}}
\epstopdfsetup{outdir=./build/}

\usepackage[inline]{enumitem}

\usepackage{tabularx}
\usepackage{booktabs}

% siunitx % cspell:disable-line
\DeclareSIUnit{\med}{med}
\DeclareSIUnit{\day}{dia}

\hyphenation{ARX ARMAX OE BJ}

\begin{document}
\bstctlcite{IEEEBST:Brazilian}

\title{Identificação de Sistema Linear por Erro de Predição}
\author{Guilherme de Paoli Beal
  \\
  {\small Universidade Federal do Rio Grande do Sul}
  \thanks{G. Beal, guilherme.beal@ufrgs.br}
}
\maketitle
\thispagestyle{empty}\pagestyle{empty}

\begin{abstract}
  Aqui vai o resumo. %% TODO
\end{abstract}

\begin{IEEEkeywords}
  Identificação por Erro de Predição, pysid, ARX, ARMAX, Output Error, Box-Jenkins
\end{IEEEkeywords}

\section{Introdução}

%% TODO desenvolver introdução
Este trabalho visa à aplicação de identificação de um sistema discreto, linear e invariante no tempo, através do método de erro de predição, ou \emph{Prediction Error Method} (PEM).
Uma batelada de dados de um sistema, supostamente desconhecido, é fornecida, em que a saída está contaminada por ruído de medição.
A identificação é realizada utilizando diferentes classes de modelo e ordens, e os resultados são comparados.

As implementações são desenvolvidas em Python, versão 3.9.12.
A identificação por erro de predição utiliza o pacote \texttt{pysid} em versão de desenvolvimento 0.1.0.
O código deste projeto está publicado em \href{https://github.com/GuiBeal/system-identification}{\texttt{github.com/GuiBeal/system-identification}}.

\section{Identificação por Erro de Predição}

Considere um sistema discreto, linear, invariante no tempo, com uma única entrada e uma única saída.
A resposta deste sistema é dada por
\begin{equation}\label{eq:system}
  y(t) = G_0(z) \Prod u(t) + \nu(t)
  ,\; t \in \mathbb{N}
  ,
\end{equation}
em que
$y(t)$ é o sinal de saída,
$G_0(z)$ é a função de transferência do sistema,
$u(t)$ é o sinal de entrada,
$\nu(t)$ é um ruído de medição desconhecido, %% TODO verificar nomenclatura
$z$ é o operador de avanço --- de modo que $z \Prod x(t) = x(t+1)$ --- e
$t \in \mathbb{N}$ é a variável de tempo discreto.

\subsection{Modelos}

O modelo busca representar o sistema real $G_0(q)$ expresso em \eqref{eq:system}.
Em particular, na identificação por erro de predição, o modelo procurar caracterizar também o ruído $\nu(t)$.

Quatro diferentes classes de modelo são exploradas neste trabalho:
\begin{itemize}
  \item Autorregressivo com Entrada Externa, ou \emph{Autoregressive with Extra Input} (ARX);
  \item Autorregressivo com Média Móvel e Entrada Externa, ou \emph{Autoregressive Moving Average with Extra Input} (ARMAX);
  \item Erro na Saída, ou \emph{Output Error} (OE); e
  \item Box-Jenkins (BJ).
\end{itemize}
Essas classes são casos particulares de um modelo mais genérico, definido por
\begin{equation}\label{eq:model}
  A(q) \Prod y(t) = \dfrac{B(q)}{F(q)} \Prod u(t) + \dfrac{C(q)}{D(q)} \Prod e(t)
  ,
\end{equation}
em que $e(t)$ é ruído branco.
Os polinômios têm os formatos
\begin{align}
  A(q) &= 1 + a_1 \Prod q^{-1} + \dotsb + a_{n_a} \Prod q^{-n_a}
  ,
  \\
  B(q) &= q^{-n_k} \Prod \left(b_0 + b_1 \Prod q^{-1} + \dotsb + b_{n_b} \Prod q^{-n_b}\right)
  ,
  \\
  C(q) &= 1 + c_1 \Prod q^{-1} + \dotsb + c_{n_c} \Prod q^{-n_c}
  ,
  \\
  D(q) &= 1 + d_1 \Prod q^{-1} + \dotsb + d_{n_d} \Prod q^{-n_d}
  , \text{ e}
  \\
  F(q) &= 1 + f_1 \Prod q^{-1} + \dotsb + f_{n_f} \Prod q^{-n_f}
  ,
\end{align}
em que $n_a$, $n_b$, $n_c$, $n_d$ e $n_f$ são as ordens dos polinômios e $n_k$ é o número de atrasos da entrada para a saída.
Definindo as funções de transferência
\begin{align}
  G(q) &= \dfrac{B(q)}{A(q) \Prod F(q)}
  , \text{ e}
  \\
  H(q) &= \dfrac{C(q)}{A(q) \Prod D(q)}
  ,
\end{align}
então \eqref{eq:model} pode ser reescrita como
\begin{equation}\label{eq:model-tf}
  y(t) = G(q) \Prod u(t) + H(q) \Prod e(t)
  .
\end{equation}
Note que $G(q)$ caracteriza o sistema real $G_0(q)$.
Por sua vez, o ruído de medição $\nu(t)$ é representado como ruído branco $e(t)$ filtrado por $H(q)$: %% TODO revisar nomenclatura
\begin{equation}
  \nu(t) = H(q) \Prod e(t)
  .
\end{equation}

Pelas definições dos polinômios,
\begin{equation}\label{eq:H-inf}
  H(\infty) = 1
  .
\end{equation}
Outrossim, se o modelo representa um sistema amostrado, então $G(q)$ deve ser estritamente própria --- isto é, o grau de seu denominador é maior que o de seu numerador --- o que é garantido com $n_k \geq 1$.

Ressalta-se que este modelo genérico pode apresentar definições e notações diferentes, como em \cite{book:Ljung1999, book:Aguirre2007, misc:matlab-polynomial-models}.
A definição aqui apresentada corresponde aos formatos dos polinômios retornados pelas funções do pacote \texttt{pysid}.


\subsection{Classes de Modelos}
As quatro classes de modelo são obtidas fixando determinadas ordens no modelo genérico.

\subsubsection{ARX}
Neste modelo, há liberdade nas escolhas de $n_a$, $n_b$ e $n_k$, enquanto $n_c = n_d = n_f = 0$.
Assim, as funções de transferência tornam-se
\begin{align}
  G(q) &= \dfrac{B(q)}{A(q)}
  , \text{ e}
  \\
  H(q) &= \dfrac{1}{A(q)}
  .
\end{align}
Esta classe tem a vantagem de resultar numa minimização linear nos parâmetros, os quais podem ser obtidos por mínimos quadrados.

\subsubsection{ARMAX}
Este modelo requer a arbitração das ordens $n_a$, $n_b$, $n_c$ e $n_k$, com $n_d = n_f = 0$.
Portanto, as funções de transferência são
\begin{align}
  G(q) &= \dfrac{B(q)}{A(q)}
  , \text{ e}
  \\
  H(q) &= \dfrac{C(q)}{A(q)}
  .
\end{align}

\subsubsection{\emph{Output Error}}
Para este modelo são escolhidas as ordens $n_b$, $n_f$ e $n_k$, fixando $n_a = n_c = n_d = 0$.
Nesse casso, tem-se
\begin{align}
  G(q) &= \dfrac{B(q)}{F(q)}
  , \text{ e}
  \\
  H(q) &= 1
  .
\end{align}
Note que este modelo considera que o ruído de medição é ruído branco. % TODO nomenclatura

\subsubsection{Box-Jenkins}
Finalmente, neste modelo arbitra-se $n_b$, $n_c$, $n_d$, $n_f$ e $n_k$, com $n_a = 0$.
Com isso, obtém-se
\begin{align}
  G(q) &= \dfrac{B(q)}{F(q)}
  , \text{ e}
  \\
  H(q) &= \dfrac{C(q)}{F(q)}
  .
\end{align}

\subsection{Predição}

A predição é realizada aplicando
\begin{equation}\label{eq:prediction}
  \hat{y}(t) = L_u(q) \Prod u(t) + L_y(q) \Prod y(t)
\end{equation}
com
\begin{align}
  L_u(q) &= \dfrac{G(q)}{H(q)}
  , \text{ e}
  \\
  L_y(q) &= 1 - \dfrac{1}{H(q)}
  .
\end{align}
Embora isso não seja diretamente evidenciado por \eqref{eq:prediction}, o fato de $G(q)$ ser estritamente própria juntamente com \eqref{eq:H-inf} garante que a predição $\hat{y}(t)$ no instante $t = t_0$ depende somente de valores de $u(t)$ e $y(t)$ em instantes $t < t_0$ --- isto é, a predição é realizado a partir de valores anteriores dos sinais de entrada e saída medidos.

O erro de predição é definido como
\begin{equation}
  \varepsilon(t)
  = y(t) - \hat{y}(t)
  = \dfrac{y(t) - G(q) \Prod u(t)}{H(q)}
  .
\end{equation}
Assim, o erro quadrático médio de predição é definido por
\begin{equation}\label{eq:mse}
  J = \dfrac{1}{N} \Prod \sum_{t=1}^{N}{\left(\varepsilon(t)\right)^2}
  ,
\end{equation}
em que $N$ é o número de amostras preditas.

\subsection{Identificação}
A identificação por erro de predição visa, a partir de um conjunto de dados de entrada $u(t)$ e de saída $y(t)$ medidos do processo, a identificar os parâmetros dos polinômios do modelo de forma a minimizar o custo expresso em \eqref{eq:mse}.
O algoritmo aplicado neste processo depende da classe do modelo e está fora do escopo deste trabalho.
Estes são implementados pelo pacote \texttt{pysid}.

\section{Critério de Informação de Akaike}

Para avaliar a qualidade dos modelos identificados será utilizado o critério de informação de Akaike, ou \emph{Akaike Information Criterion} (AIC).
Este critério pondera a qualidade de predição junto à ordem do modelo, penalizando critérios de maior ordem.
Considerando o erro de predição, o critério é definido por
\begin{equation}
  \text{AIC} = N \Prod \log\left(J\right) + 2 \Prod k
  ,
\end{equation}
em que $N$ é o número de amostras preditas, $k$ é o número de parâmetros do modelo e $J$ é o erro médio quadrático de predição definido em \eqref{eq:mse} \cite{book:Soderstrom1989}.

Para conjuntos com poucas amostras, pode-se utilizar um critério adaptado, definido por
\begin{equation}\label{eq:akaike-small}
  \text{AICc} = \text{AIC} + \dfrac{2 \Prod k \Prod \left(k + 1\right)}{N - k - 1}
  .
\end{equation}
Note que $N \to \infty \implies \text{AICc} \to \text{AIC}$ --- isto é, o critério adaptado se aproximado do critério original à medida que cresce o número de amostras considerado.

\section{Dados}

Os dados são apresentados na Figura \ref{fig:data}, em que o sinal $u(t)$ é multiplicado por um fator de $10$ para melhor visualização.
Ambos os sinais $u(t)$ e $y(t)$ contêm 200 amostras.
Note que a entrada $u(t)$ é similar a uma onda quadrada.
\begin{figure}[!htbp]
  \centering
  \includegraphics[width=\linewidth]{data_folded}
  \caption{Dados de entrada e saída.}
  \label{fig:data}
\end{figure}

Na Figura \ref{fig:data}, a linha tracejada vertical divide os dados em dois conjuntos.
Para a identificação dos modelos, utilizam-se os dados à esquerda desta linha, correspondentes às primeiras 160 amostras.
As 40 amostras restantes são aplicadas na validação dos modelos identificados.

A Figura \ref{fig:data_fourier} mostra o espectro de amplitude dos dados --- isto é, a magnitude da transformada de Fourier dos sinais --- em escala logarítmica.
Note como ambos os sinais possuem um conteúdo elevado em baixas frequências, seguidos de picos pontuais que decrescem com o aumento da frequência.
\begin{figure}[!htbp]
  \centering
  \includegraphics[width=\linewidth]{data_fourier_log}
  \caption{Espectro de amplitude dos dados de entrada e saída.}
  \label{fig:data_fourier}
\end{figure}

\section{Resultados}

Diversas identificações são realizadas, variando as ordens dos polinômios e o atraso entre as quatro classes de modelos.
A Tabela \ref{tab:orders} sintetiza os intervalos de variação desses valores para cada modelo.
% As ordens $n_a$, $n_d$ e $n_f$ variam a partir de 1.
% Note que, no modelo ARMAX, $n_c = 0$ equivaleria ao modelo ARX; portanto, essa ordem nesse modelo varia a partir de 1.
A fim de considerar sistemas amostrados, todos os sistemas tem atraso de pelo menos um amostra --- isto é, $n_k \geq 1$.

\begin{table}[!htbp]
  \centering
  \caption{Intervalo de variação das ordens dos polinômios}
  \label{tab:orders}
  \newcolumntype{C}{>{\centering\arraybackslash}X}
  \setlength{\extrarowheight}{1pt}
  \begin{tabularx}{\linewidth}{@{} c *{6}{C} c @{}}
    \toprule
    Classe & $n_a$    & $n_b$    & $n_c$    & $n_d$    & $n_f$    & $n_k$    & Total \\
    \midrule
    ARX    & $[1, 4]$ & $[0, 4]$ & --       & --       & --       & $[1, 4]$ & 80    \\
    ARMAX  & $[1, 4]$ & $[0, 4]$ & $[1, 4]$ & --       & --       & $[1, 4]$ & 320   \\
    OE     & --       & $[0, 4]$ & --       & --       & $[1, 4]$ & $[1, 4]$ & 80    \\
    BJ     & --       & $[0, 4]$ & $[0, 4]$ & $[1, 4]$ & $[1, 4]$ & $[1, 4]$ & 1600  \\
    \midrule
    Total  &          &          &          &          &          &          & 2080  \\
    \bottomrule
  \end{tabularx}
\end{table}

É importante ressaltar que dentre as 1600 identificações realizadas com a classe BJ 960 apresentaram algum problema de implementação interno ao pacote \texttt{pysid}.
Nesses casos, esses modelos foram simplesmente descartados.

Para cada identificação são calculados o erro quadrático médio, conforme \eqref{eq:mse}, e o critério de informação de Akaike para pequenos conjuntos de dados, conforme \eqref{eq:akaike-small}.
Ambos são realizados tanto para o conjunto de dados de identificação --- denotado pelo subíndice $i$ --- como para o conjunto de validação --- denotado pelo subíndice $v$.

Para cada uma das quatro classes, a Tabela \ref{tab:results-class} apresenta os seis melhores modelos resultantes, com base no critério de informação de Akaike com os dados de validação.
Os demais custos também são
Os modelos são unicamente identificados por um índice, exibidos na primeira coluna.

\begin{table*}
  \caption{Melhores resultados por classe com base no critério $\text{AICc}_v$}
  \label{tab:results-class}
  \newcolumntype{C}{>{\centering\arraybackslash}X}
  \setlength{\extrarowheight}{1pt}
  \begin{tabularx}{\textwidth}{@{} c *{10}{C} c @{}}
    \toprule
    \#   & Classe & $n_a$   & $n_b$   & $n_c$   & $n_d$   & $n_f$   & $n_k$   & $\text{AICc}_v$ & $\text{AICc}_i$ & $J_v$         & $J_i$        \\
    \midrule
    24   & ARX    & \num{2} & \num{1} & --      & --      & --      & \num{1} & \num{79.463 }   & \num{288.142}   & \num{5.801  } & \num{5.750 } \\
    44   & ARX    & \num{3} & \num{1} & --      & --      & --      & \num{1} & \num{80.358 }   & \num{289.971}   & \num{5.556  } & \num{5.740 } \\
    28   & ARX    & \num{2} & \num{2} & --      & --      & --      & \num{1} & \num{80.596 }   & \num{276.503}   & \num{5.589  } & \num{5.276 } \\
    20   & ARX    & \num{2} & \num{0} & --      & --      & --      & \num{1} & \num{80.820 }   & \num{290.423}   & \num{6.384  } & \num{5.910 } \\
    40   & ARX    & \num{3} & \num{0} & --      & --      & --      & \num{1} & \num{81.303 }   & \num{291.572}   & \num{6.074  } & \num{5.875 } \\
    32   & ARX    & \num{2} & \num{3} & --      & --      & --      & \num{1} & \num{82.077 }   & \num{278.289}   & \num{5.410  } & \num{5.264 } \\
    \midrule
    160  & ARMAX  & \num{2} & \num{0} & \num{1} & --      & --      & \num{1} & \num{82.406 }   & \num{292.387}   & \num{6.244  } & \num{5.905 } \\
    240  & ARMAX  & \num{3} & \num{0} & \num{1} & --      & --      & \num{1} & \num{82.440 }   & \num{291.698}   & \num{5.853  } & \num{5.802 } \\
    192  & ARMAX  & \num{2} & \num{2} & \num{1} & --      & --      & \num{1} & \num{83.302 }   & \num{278.685}   & \num{5.579  } & \num{5.277 } \\
    144  & ARMAX  & \num{1} & \num{4} & \num{1} & --      & --      & \num{1} & \num{83.435 }   & \num{299.330}   & \num{5.199  } & \num{5.922 } \\
    112  & ARMAX  & \num{1} & \num{2} & \num{1} & --      & --      & \num{1} & \num{83.574 }   & \num{294.450}   & \num{6.021  } & \num{5.902 } \\
    336  & ARMAX  & \num{4} & \num{1} & \num{1} & --      & --      & \num{1} & \num{84.031 }   & \num{282.669}   & \num{5.277  } & \num{5.337 } \\
    \midrule
    465  & OE     & --      & \num{4} & --      & --      & \num{1} & \num{2} & \num{203.538}   & \num{635.576}   & \num{112.710} & \num{49.103} \\
    469  & OE     & --      & \num{4} & --      & --      & \num{2} & \num{2} & \num{205.126}   & \num{637.238}   & \num{108.924} & \num{48.942} \\
    466  & OE     & --      & \num{4} & --      & --      & \num{1} & \num{3} & \num{208.209}   & \num{688.227}   & \num{126.671} & \num{68.237} \\
    468  & OE     & --      & \num{4} & --      & --      & \num{2} & \num{1} & \num{210.717}   & \num{620.915}   & \num{125.264} & \num{44.195} \\
    470  & OE     & --      & \num{4} & --      & --      & \num{2} & \num{3} & \num{211.195}   & \num{690.408}   & \num{126.771} & \num{68.234} \\
    464  & OE     & --      & \num{4} & --      & --      & \num{1} & \num{1} & \num{211.208}   & \num{622.115}   & \num{136.532} & \num{45.141} \\
    \midrule
    1136 & BJ     & --      & \num{2} & \num{0} & \num{2} & \num{1} & \num{1} & \num{82.909 }   & \num{276.778}   & \num{5.524  } & \num{5.214 } \\
    816  & BJ     & --      & \num{1} & \num{0} & \num{2} & \num{1} & \num{1} & \num{83.040 }   & \num{273.646}   & \num{5.941  } & \num{5.183 } \\
    832  & BJ     & --      & \num{1} & \num{0} & \num{3} & \num{1} & \num{1} & \num{84.041 }   & \num{276.120}   & \num{5.683  } & \num{5.193 } \\
    1456 & BJ     & --      & \num{3} & \num{0} & \num{2} & \num{1} & \num{1} & \num{84.939 }   & \num{277.505}   & \num{5.398  } & \num{5.167 } \\
    1140 & BJ     & --      & \num{2} & \num{0} & \num{2} & \num{2} & \num{1} & \num{85.844 }   & \num{278.965}   & \num{5.521  } & \num{5.214 } \\
    848  & BJ     & --      & \num{1} & \num{0} & \num{4} & \num{1} & \num{1} & \num{86.957 }   & \num{274.874}   & \num{5.677  } & \num{5.083 } \\
    \bottomrule
  \end{tabularx}
\end{table*}

Comparando a ordem de grandeza de cada critério de qualidade na Tabela \ref{tab:results-class}, nota-se que os modelos ARX, ARMAX e BJ apresentam um desempenho consideravelmente melhor que os modelos OE;
isto é um indicativo de que o ruído de medição não é branco, conforme supõe esse último modelo.
Além disso, para os modelos ARX, ARMAX e BJ, todos as identificações presentes na Tabela \ref{tab:results-class} têm $n_k = 1$ --- isto é, um atraso da entrada para a saída de apenas uma amostra.
Destaca-se, ainda, que nos modelos ARMAX e BJ, constam na Tabela \ref{tab:results-class} apenas identificações com $n_c$ partindo de seu valor mínimo --- 1 para ARMAX e 0 para BJ ---, levando à interpretação de que o numerador de $H(q)$ tem ordem pequena.

A Tabela \ref{tab:results} mostra as 20 melhores identificações, classificadas pelo critério de informação de Akaike adaptado com os dados de validação.
Note a prevalência de modelos ARX, bem como das características discutidas no parágrafo anterior.

\begin{table*}
  \caption{Melhores resultados gerais com base no critério $\text{AICc}_v$}
  \label{tab:results}
  \newcolumntype{C}{>{\centering\arraybackslash}X}
  \setlength{\extrarowheight}{1pt}
  \begin{tabularx}{\textwidth}{@{} c *{10}{C} c @{}}
    \toprule
    \#   & Classe & $n_a$   & $n_b$   & $n_c$   & $n_d$   & $n_f$   & $n_k$   & $\text{AICc}_v$ & $\text{AICc}_i$ & $J_v$  & $J_i$       \\
    24   & ARX    & \num{2} & \num{1} &  --     &  --       &  --   & \num{1} & \num{79.463} & \num{288.142} & \num{5.801} & \num{5.75}  \\
    44   & ARX    & \num{3} & \num{1} &  --     &  --       &  --   & \num{1} & \num{80.358} & \num{289.971} & \num{5.556} & \num{5.74}  \\
    28   & ARX    & \num{2} & \num{2} &  --     &  --       &  --   & \num{1} & \num{80.596} & \num{276.503} & \num{5.589} & \num{5.276} \\
    20   & ARX    & \num{2} & \num{0} &  --     &  --       &  --   & \num{1} & \num{80.82 } & \num{290.423} & \num{6.384} & \num{5.91}  \\
    40   & ARX    & \num{3} & \num{0} &  --     &  --       &  --   & \num{1} & \num{81.303} & \num{291.572} & \num{6.074} & \num{5.875} \\
    32   & ARX    & \num{2} & \num{3} &  --     &  --       &  --   & \num{1} & \num{82.077} & \num{278.289} & \num{5.41 } & \num{5.264} \\
    64   & ARX    & \num{4} & \num{1} &  --     &  --       &  --   & \num{1} & \num{82.145} & \num{288.495} & \num{5.419} & \num{5.611} \\
    160  & ARMAX  & \num{2} & \num{0} & \num{1} &  --       &  --   & \num{1} & \num{82.406} & \num{292.387} & \num{6.244} & \num{5.905} \\
    240  & ARMAX  & \num{3} & \num{0} & \num{1} &  --       &  --   & \num{1} & \num{82.44 } & \num{291.698} & \num{5.853} & \num{5.802} \\
    1136 & BJ     &  --     & \num{2} & \num{0} & \num{2} & \num{1} & \num{1} & \num{82.909} & \num{276.778} & \num{5.524} & \num{5.214} \\
    816  & BJ     &  --     & \num{1} & \num{0} & \num{2} & \num{1} & \num{1} & \num{83.04 } & \num{273.646} & \num{5.941} & \num{5.183} \\
    36   & ARX    & \num{2} & \num{4} &  --     &  --       &  --   & \num{1} & \num{83.283} & \num{281.484} & \num{5.179} & \num{5.297} \\
    192  & ARMAX  & \num{2} & \num{2} & \num{1} &  --       &  --   & \num{1} & \num{83.302} & \num{278.685} & \num{5.579} & \num{5.277} \\
    60   & ARX    & \num{4} & \num{0} &  --     &  --       &  --   & \num{1} & \num{83.403} & \num{291.217} & \num{5.995} & \num{5.784} \\
    144  & ARMAX  & \num{1} & \num{4} & \num{1} &  --       &  --   & \num{1} & \num{83.435} & \num{299.33 } & \num{5.199} & \num{5.922} \\
    112  & ARMAX  & \num{1} & \num{2} & \num{1} &  --       &  --   & \num{1} & \num{83.574} & \num{294.45 } & \num{6.021} & \num{5.902} \\
    48   & ARX    & \num{3} & \num{2} &  --     &  --       &  --   & \num{1} & \num{83.698} & \num{278.625} & \num{5.634} & \num{5.275} \\
    336  & ARMAX  & \num{4} & \num{1} & \num{1} &  --       &  --   & \num{1} & \num{84.031} & \num{282.669} & \num{5.277} & \num{5.337} \\
    832  & BJ     &  --     & \num{1} & \num{0} & \num{3} & \num{1} & \num{1} & \num{84.041} & \num{276.12 } & \num{5.683} & \num{5.193} \\
    52   & ARX    & \num{3} & \num{3} &  --     &  --       &  --   & \num{1} & \num{84.774} & \num{280.442} & \num{5.376} & \num{5.263} \\
    \bottomrule
  \end{tabularx}
\end{table*}

A partir das identificações, são obtidas as funções de transferência do sistema $G(q)$ e do ruído $H(q)$.
A fim de compará-las visualmente, as Figuras \ref{fig:bode-system} e \ref{fig:bode-noise} apresentam suas respostas em frequência, considerando as identificações na Tabela \ref{tab:results}.

\begin{figure}[!htbp]
  \centering
  \includegraphics[width=\linewidth]{bode_G_AICCv}
  \caption{Resposta em frequências de $G(q)$ a partir das identificações na Tabela \ref{tab:results}.}
  \label{fig:bode-system}
\end{figure}

\begin{figure}[!htbp]
  \centering
  \includegraphics[width=\linewidth]{bode_H_AICCv}
  \caption{Resposta em frequências de $H(q)$ a partir das identificações na Tabela \ref{tab:results}.}
  \label{fig:bode-noise}
\end{figure}

Note, através da Figura \ref{fig:bode-system}, que, entre todas as identificações, o comportamento do sistema $G(q)$ é semelhante nas baixas frequências mas discordante nas altas frequências.
Isso está em consonância com o fato de que os dados trazem mais informações nas baixas frequências, conforme visto na Figura \ref{fig:data_fourier}.
Por outro lado, a Figura \ref{fig:bode-noise} mostra que o sistema $H(q)$ é mais diverso entre as diferentes identificações.
% A identificação do ruído não é tão boa quanto a identificação do sistema.

\section{Conclusões}

\bibliographystyle{IEEEtran}
\bibliography{bib/book, bib/misc, bib/controlIEEE}

\end{document}
