\documentclass{ppgeesa}

\usepackage[hidelinks]{hyperref}
\usepackage{url}
\usepackage{breakurl}

\usepackage[detect-weight,detect-family]{siunitx}
\sisetup{output-decimal-marker = {,}}
\sisetup{exponent-product = {\cdot}}

\usepackage{Math}
\mathtoolsset{showonlyrefs}
\usepackage{resizegather}
\usepackage{icomma}

\usepackage{Figure}
% \graphicspath{{../../../Python/ELE410_Aprendizado_Supervisionado_de_Modelos_Parametricos/T1/fig/}}

\usepackage[inline]{enumitem}

% siunitx % cspell:disable-line
\DeclareSIUnit{\med}{med}
\DeclareSIUnit{\day}{dia}

\begin{document}
\bstctlcite{IEEEBST:Brazilian}

\title{Identificação de Sistema Linear por Erro de Predição}
\author{Guilherme de Paoli Beal
  \\
  {\small Universidade Federal do Rio Grande do Sul}
  \thanks{G. Beal, guilherme.beal@ufrgs.br}
}
\maketitle
\thispagestyle{empty}\pagestyle{empty}

\begin{abstract}
  Aqui vai o resumo.
\end{abstract}

\begin{IEEEkeywords}
  Identificação, Sistema Linear
\end{IEEEkeywords}

\section{Introdução}

O código desenvolvido para este projeto está publicado em \href{https://github.com/GuiBeal/system-identification}{\texttt{github.com/GuiBeal/system\-identification}}.

\section{Dados}

\section{Conclusões}

% \bibliographystyle{IEEEtran}
% \bibliography{articles,books,electronic,controlIEEE}

\end{document}
